
% This is "sig-alternate.tex" V2.0 May 2012
% This file should be compiled with V2.5 of "sig-alternate.cls" May 2012
%
% This example file demonstrates the use of the 'sig-alternate.cls'
% V2.5 LaTeX2e document class file. It is for those submitting
% articles to ACM Conference Proceedings WHO DO NOT WISH TO
% STRICTLY ADHERE TO THE SIGS (PUBS-BOARD-ENDORSED) STYLE.
% The 'sig-alternate.cls' file will produce a similar-looking,
% albeit, 'tighter' paper resulting in, invariably, fewer pages.
%
% ----------------------------------------------------------------------------------------------------------------
% This .tex file (and associated .cls V2.5) produces:
%       1) The Permission Statement
%       2) The Conference (location) Info information
%       3) The Copyright Line with ACM data
%       4) NO page numbers
%
% as against the acm_proc_article-sp.cls file which
% DOES NOT produce 1) thru' 3) above.
%
% Using 'sig-alternate.cls' you have control, however, from within
% the source .tex file, over both the CopyrightYear
% (defaulted to 200X) and the ACM Copyright Data
% (defaulted to X-XXXXX-XX-X/XX/XX).
% e.g.
% \CopyrightYear{2007} will cause 2007 to appear in the copyright line.
% \crdata{0-12345-67-8/90/12} will cause 0-12345-67-8/90/12 to appear in the copyright line.
%
% ---------------------------------------------------------------------------------------------------------------
% This .tex source is an example which *does* use
% the .bib file (from which the .bbl file % is produced).
% REMEMBER HOWEVER: After having produced the .bbl file,
% and prior to final submission, you *NEED* to 'insert'
% your .bbl file into your source .tex file so as to provide
% ONE 'self-contained' source file.
%
% ================= IF YOU HAVE QUESTIONS =======================
% Questions regarding the SIGS styles, SIGS policies and
% procedures, Conferences etc. should be sent to
% Adrienne Griscti (griscti@acm.org)
%
% Technical questions _only_ to
% Gerald Murray (murray@hq.acm.org)
% ===============================================================
%
% For tracking purposes - this is V2.0 - May 2012

\documentclass{IEEEtran}

% Load basic packages
\usepackage{balance}  % to better equalize the last page
\usepackage{graphics} % for EPS, load graphicx instead
%\usepackage[T1]{fontenc}
%\usepackage{txfonts}
%\usepackage{times}    % comment if you want LaTeX's default font
\usepackage[pdftex]{hyperref}
% \usepackage{url}      % llt: nicely formatted URLs
%\usepackage{color}
%\usepackage{textcomp}
\usepackage{booktabs}
%\usepackage{ccicons}
\usepackage{todonotes}
 \usepackage{booktabs}
 \usepackage{multirow}
 % allows for temporary adjustment of side margins
 \usepackage{chngpage}
 \usepackage{enumerate}

 % provides filler text
 \usepackage{lipsum}

\newcommand{\alexey}[1]{\textcolor{red}{{\it [Alexey says: #1]}}}
\newcommand{\maryi}[1]{\textcolor{purple}{{\it [Maryi says: #1]}}}
\newcommand{\edits}[1]{\textcolor{blue}{{\it [Edit: #1]}}}
\begin{document}

% The following command makes sure text doesn't go outside of the page limit - added by Alexey
\sloppy


%
% --- Author Metadata here ---
%\conferenceinfo{ICSE}{'16, May 14 -- 22, 2016, Austin, TX, USA}
%\CopyrightYear{2007} % Allows default copyright year (20XX) to be over-ridden - IF NEED BE.
%\crdata{0-12345-67-8/90/01}  % Allows default copyright data (0-89791-88-6/97/05) to be over-ridden - IF NEED BE.
% --- End of Author Metadata ---
% Title 1.0: Understanding the User Role in the Development of an Open Source Project using the Theory of Regulation

%\title{Understanding User Role in the Development of an Open Source Project using the Theory of Regulation}
\title{Using the Model of Regulation to Understand\\User-Developer Collaboration in a Software Project}
%
% You need the command \numberofauthors to handle the 'placement
% and alignment' of the authors beneath the title.
%
% For aesthetic reasons, we recommend 'three authors at a time'
% i.e. three 'name/affiliation blocks' be placed beneath the title.
%
% NOTE: You are NOT restricted in how many 'rows' of
% "name/affiliations" may appear. We just ask that you restrict
% the number of 'columns' to three.
%
% Because of the available 'opening page real-estate'
% we ask you to refrain from putting more than six authors
% (two rows with three columns) beneath the article title.
% More than six makes the first-page appear very cluttered indeed.
%
% Use the \alignauthor commands to handle the names
% and affiliations for an 'aesthetic maximum' of six authors.
% Add names, affiliations, addresses for
% the seventh etc. author(s) as the argument for the
% \additionalauthors command.
% These 'additional authors' will be output/set for you
% without further effort on your part as the last section in
% the body of your article BEFORE References or any Appendices.

%\numberofauthors{1} %  in this sample file, there are a *total*
% of EIGHT authors. SIX appear on the 'first-page' (for formatting
% reasons) and the remaining two appear in the \additionalauthors section.

\author{
  \alignauthor{Maryi Arciniegas-Mendez, Alexey Zagalsky, Margaret-Anne Storey, Allyson F. Hadwin\\
    \affaddr{University of Victoria}\\
    \affaddr{Victoria, BC, Canada}\\
    \email{\{maryia, alexeyza, mstorey, hadwin\}@uvic.ca}}\\
}

% \author{
%   \alignauthor{Maryi Arciniegas-Mendez\\
%     \affaddr{University of Victoria}\\
%     \affaddr{Victoria, BC, Canada}\\
%     \email{maryia@uvic.ca}}\\
%   \alignauthor{Alexey Zagalsky\\
%     \affaddr{University of Victoria}\\
%     \affaddr{Victoria, BC, Canada}\\
%     \email{alexeyza@uvic.ca}}\\
%   \alignauthor{Margaret-Anne Storey\\
%     \affaddr{University of Victoria}\\
%     \affaddr{Victoria, BC, Canada}\\
%     \email{mstorey@uvic.ca}}\\
% \and
%   \alignauthor{Allyson F. Hadwin\\
%     \affaddr{University of Victoria}\\
%     \affaddr{Victoria, BC, Canada}\\
%     \email{hadwin@uvic.ca}}\\
% }

\maketitle
\begin{abstract}

Collaboration is a complex activity that relies on awareness, communication, and coordination, and has become an integral aspect of software engineering. The widespread availability and adoption of social channels has led to a culture where modern developers participate and collaborate more frequently with one another.
While collaboration in software engineering has been studied extensively, gaining an understanding of collaboration in today's participatory development culture is challenging as there is an ever-growing number of stakeholders and feature needs evolve rapidly when software is continuously deployed.

In this paper, we borrow and adapt a Model of Regulation from the learning science domain and customize it so that it can be used to describe and understand regulation processes in  modern collaborative development projects.
Furthermore, we use the model as a lens to gain insights into how users regulate an open source development project.
Finally, we discuss how this Model of Regulation shows potential for revealing how well certain tools and channels work for supporting regulation in software development, and to recognize communication breakdowns and gaps.

%Peggy: saving old version here in case my version is terrible...
%Collaboration has become an integral aspect of software engineering. The widespread availability and adoption of social channels has led to a culture where today's developers participate and collaborate more frequently with one another. However, capturing and understanding the collaborative nature of software development is challenging. To address the shortcomings of the existing models\todo[inline]{What models are those?}, we propose borrowing, adapting, and extending educational psychology's Model of Regulation to software engineering.
%
%In this paper we demonstrate that indeed it is possible to adapt the Model of Regulation and extend it to software engineering. Furthermore, we apply the model through a case study to investigate how regulation occurs in software development, and how users participate in the regulation. \alexey{``add here a paragraph or two about specific findings of this study.''}

% Peggy: suggest to add this to conclusions and to abstract... (copied from intro)
%The framework acts as a lens allowing to understand the specific types of regulation between users and project contributors. The framework helps us to identify where certain tools and channels work well, and to recognize communication breakdowns and gaps.

% Thanks to new platforms and technologies, the community of software developers has experienced a growing participatory culture in collaborative work at a scale and speed never seen before. Computer-based tools have reduced distance barriers facilitating more opportunities to collaborate. Awareness has been identify as a key feature for tools that support this wide collaboration but what awareness entails and what information should be the focused are open questions. Moreover, challenges in collaboration are still present despite the use of multiple tools with different awareness perspectives. Thus, suggesting that collaboration has processes involved not yet supported by the combined use of tools. In this paper, we extend our work with the Model of regulation by using it to evaluate computer-based tools that support the collaboration of an open-source development project. The results illustrate individual and collective support for regulation on collaboration. We discuss how processes not supported by the tool (or not supported by the group use) can be the cause for current issues in collaboration. Finally, we discuss limitations of the model as a resource to evaluate tool support and provide some directions for future research.
\end{abstract}

% A category with the (minimum) three required fields
\category{H.5.3}{Group and Organization Interfaces}{Computer-Supported Collaborative Work}
%A category including the fourth, optional field follows...
%\category{D.2.8}{Software Engineering}{Metrics}[complexity measures, performance measures]

\terms{Human Factors}

\keywords{Theory, Collaboration, Regulation, Open-Software Development, User Participation}


%ACKNOWLEDGMENTS are optional
%\section{Acknowledgments}
%The authors would like to thank Cassandra Petrachenko for editing support and insightful comments that contributed to this work. We also thank Carlene Lebeuf and Maria Ferman for their assistance and help with the data collection and analysis processes.

%\newpage

%
% The following two commands are all you need in the
% initial runs of your .tex file to
% produce the bibliography for the citations in your paper.
\bibliographystyle{abbrv}
\bibliography{Bibliotek}  % sigproc.bib is the name of the Bibliography in this case
% You must have a proper ".bib" file
%  and remember to run:
% latex bibtex latex latex
% to resolve all references
%
% ACM needs 'a single self-contained file'!
%
%APPENDICES are optional
\balance
\end{document}
